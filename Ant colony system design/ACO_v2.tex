%% 
%% Copyright 2007, 2008, 2009 Elsevier Ltd
%% 
%% This file is part of the 'Elsarticle Bundle'.
%% ---------------------------------------------
%% 
%% It may be distributed under the conditions of the LaTeX Project Public
%% License, either version 1.2 of this license or (at your option) any
%% later version.  The latest version of this license is in
%%    http://www.latex-project.org/lppl.txt
%% and version 1.2 or later is part of all distributions of LaTeX
%% version 1999/12/01 or later.
%% 
%% The list of all files belonging to the 'Elsarticle Bundle' is
%% given in the file `manifest.txt'.
%% 

%% Template article for Elsevier's document class `elsarticle'
%% with numbered style bibliographic references
%% SP 2008/03/01

\documentclass[preprint,12pt]{elsarticle}

%% Use the option review to obtain double line spacing
%% \documentclass[authoryear,preprint,review,12pt]{elsarticle}

%% Use the options 1p,twocolumn; 3p; 3p,twocolumn; 5p; or 5p,twocolumn
%% for a journal layout:
%% \documentclass[final,1p,times]{elsarticle}
%%\documentclass[final,1p,times,twocolumn]{elsarticle}
%% \documentclass[final,3p,times]{elsarticle}
%% \documentclass[final,3p,times,twocolumn]{elsarticle}
%% \documentclass[final,5p,times]{elsarticle}
%% \documentclass[final,5p,times,twocolumn]{elsarticle}

%% For including figures, graphicx.sty has been loaded in
%% elsarticle.cls. If you prefer to use the old commands
%% please give \usepackage{epsfig}

%% The amssymb package provides various useful mathematical symbols
\usepackage{amssymb}
%% The amsthm package provides extended theorem environments
\usepackage{amsthm}

\usepackage{amsmath}

\usepackage[font=small,skip=3pt]{subcaption}
\usepackage{natbib}  

%% The lineno packages adds line numbers. Start line numbering with
%% \begin{linenumbers}, end it with \end{linenumbers}. Or switch it on
%% for the whole article with \linenumbers.
%% \usepackage{lineno}

\journal{Ocean Engineering}

\begin{document}

\begin{frontmatter}

%% Title, authors and addresses

%% use the tnoteref command within \title for footnotes;
%% use the tnotetext command for theassociated footnote;
%% use the fnref command within \author or \address for footnotes;
%% use the fntext command for theassociated footnote;
%% use the corref command within \author for corresponding author footnotes;
%% use the cortext command for theassociated footnote;
%% use the ead command for the email address,
%% and the form \ead[url] for the home page:
%% \title{Title\tnoteref{label1}}
%% \tnotetext[label1]{}
%% \author{Name\corref{cor1}\fnref{label2}}
%% \ead{email address}
%% \ead[url]{home page}
%% \fntext[label2]{}
%% \cortext[cor1]{}
%% \address{Address\fnref{label3}}
%% \fntext[label3]{}

\title{Ant colony optimization method for bi-objective distribution system design space exploration}

%% use optional labels to link authors explicitly to addresses:
%% \author[label1,label2]{}
%% \address[label1]{}
%% \address[label2]{}

\author[1]{Colin P. F. Shields}
\author[1]{Michael J. Sypniewski} 
\author[1]{David J. Singer} 
\address[1]{University of Michigan, United States}

\begin{abstract}
%% Text of abstract
ACO

\end{abstract}

\begin{keyword}
%% keywords here, in the form: keyword \sep keyword

%% PACS codes here, in the form: \PACS code \sep code

%% MSC codes here, in the form: \MSC code \sep code
%% or \MSC[2008] code \sep code (2000 is the default)
System architecture \sep network theory \sep early-stage ship design

\end{keyword}

\end{frontmatter}

%% \linenumbers

%% main text
\section{Introduction} \label{sec:intro}
Distribution system design is often postponed until detail design and construction due to the high amount of design detail traditionally required to accurately evaluate system performance against high-level requirements such as survivability.
Historically this approach has led to poor design outcomes since potential design drivers are only considered after the design has been sufficiently constrained to inhibit efficient and effective design changes.
In this paper, we suggest a network-based optimization analysis that addresses high-level requirements and accommodates the level of information available in early stage ship design.
\\

%Current methods for resolving this lack of design detail in early-stage design are either incomplete or self-constraining.
%
%head pressure, blah blah blah, intractable for ESSD. people assume we need a lot of design detail for distribution system analysis.
%
%Ship design requires appropriate analysis to make well informed decisions.
%
%Often times in ship design there is an insufficient amount of information available to conduct a required analysis. 
%
%As a product of traditional engineering disciplines and 
%
%Why is ship design complex? Why are decisions hard? Uncertainty of future.
%
%Ship design is a complex process that requires analysis to make well-informed decisions. To conduct an analysis, information is needed to be known able the current state of the design.
%
%Ship design is a complex process during which decisions must be made to ensure progress.
%
%Design decisions are justified by conducting analyses. These analyses range from being relatively simple to complex.
%
%decisions made in early stage design dictate high level functionality
%
%design are evaluated against one or more objective functions
%
%Ship design is a complex process throughout which decisions are justified by engineering analysis. We run into issues when the analysis requires more information than is within the current scope of the design. We have two common way of handling this lack of information: parameterization or CAD. We use parameterization as a simplification to glean information from similar ships (in a regression based approach). A downfall of this method is lack of originality, not understanding the reasons behind the other ship's decisions (check Colin's other paper for the reasons). We use CAD as a method to establish geometric relationships between components of the design but in doing so significantly constrain our design space (without the analysis to justify why this is done).
%
%Early stage decisions in ship design greatly impact the 
%
%Decisions made in early stage ship design have been 
%
%tell them what we are going to do, why they should care, and outline how we are going to do it
%
%Design decisions can be justified and understood without
%
%Understand design space
%
%How can we conduct an analysis to make informed decisions with the limited information of ESSD without assuming the pitfalls of current methods.
\begin{itemize}
	\item[Notes:]
	\item Where does the story take place? What do you need to understand about the situation to follow? What is the larger problem we are addressing?
	\begin{itemize}
		\item this is a story about marine design.
		\item the reader needs to understand:
		\begin{itemize}
			\item we make decisions based on the results of engineering analysis
			\item decision are made starting in ESSD
			\item the types of analyses being conducted is not always compatible with the amount of information available.
			\item in ESSD, there is a lack of information
			\item this lack of information is handled two ways: parameterization and CAD
			\item parameterization is a simplification that builds off of known designs so as to project knowledge to the current design
			\item post-CAD analysis makes many implicit assumptions along the way. these assumptions are decisions then that have not been vetted by any analysis.
		\end{itemize} 
		\item the story takes place within marine design, with emphasis placed on early stage design so as to combat the density of information needed to conduct an accurate analysis. more importantly though, early stage design is where most of the cost is locked in (traditionally) and the decisions are made that impact the future design of the vessel.
		\item we need to know that: marine design is complex, early-stage analysis currently requires simplification (parameterization, which eliminates interdependencies through margin, regression, and over-simplification) or more constraints/detail (fully enumerated CAD model) because we can't handle interdependencies well, decisions are made based on metrics that span disciplinary groups such that much detail/time/effort needs to be expended before we can even tell if a design is feasible
		\item the constraints put in place often result in over-constraint (i.e. in order to analysis something, we will constrain the design far more than is needed)
		\item (how do we highlight distribution systems here? how is distribution system design and optimization handled currently?)
		\item traditional optimization involves above parameterization or CAD along with optimizing interdependent systems independently from each other
		\item how do we explore a multi-system design space?
	\end{itemize}
	\item What do we need to accomplish? What specific question do we hope to answer?
	\begin{itemize}
		\item we need to create a method to analyze ship designs without over-simplification or requiring on unrealistic/constraining amount of detail
		\item this analysis requires us to be able to model and account for system interdependencies in early stage ship design
	\end{itemize}
	\item What happens to address this challenge (what did we do)?
	\begin{itemize}
		\item we propose using a network representation to handle lack of design detail (detailed in Colin's other paper), and an ACO to develop distribution systems in tandem with complexity and survivability objectives to understand design space trade-offs and responses. again, the point of exploration is not achieving an answer, it is understanding implications of decisions such that we can make informed ones
	\end{itemize}
\end{itemize}
\begin{itemize}
	\item[Plan of attack:]
	\item identify the problem that drives the research
	\item introduce the characters
	\item target an audience
\end{itemize}

%Focus on distribution system design, introduce optimizing distribution system of systems
%
%Distribution systems:
%Importance to ship design
%Difficulty
%Approaches
%
%Challenge - How to explore a multi-system design space

\section{Background}
This section introduces network representations of vessel and system of systems, a system of systems survivability metric, and ant colony optimization methods.

\subsection{Network Representations of Vessels and Systems} 
Generation and analysis methods for distribution systems often use network representations of systems and the physical vessel \cite{Gillespie2013,Rigterink2014,Dellsy2015,Trapp2015}. Networks, or graphs, are a collection of objects, called nodes, and the connections between them, called edges \cite{Newman2003}. Generally, networks are mathematically defined by their adjacency matrix $A$. The adjacency matrix is an $[n\times n]$ matrix where $n$ is the number of nodes in the network. The elements of $A$ denote the network edges: $A_{ij}=1$ if an edge exists from node $i$ to $j$, otherwise $A_{ij}=0$.

In this paper, we use two network representations, one for vessel configuration and the other for systems. These networks allow the consideration of interactions between system elements and the physical vessel without requiring geometric CAD modeling. The proposed method optimizes the network representing the system to explore how interactions between system components in the physical vessel affect system performance. This makes the proposed approach applicable to the earlier stages of ship design when design detail required for CAD modeling is not available. Further, network representations can scale with design fidelity \cite{Rigterink2014, Gillespie2012} meaning the method can be used throughout the design process.

\subsubsection{Vessel Configuration Network} \label{sec:vesselmodel}
Vessels can be represented as a network $V$ by defining modeling volumes and their adjacencies in physical configuration. Shown in Figure (here), the vessel is first divided into volumes, for instance compartments or structural zones. These volumes become nodes in $V$ and adjacencies between volumes are edges. 

The configuration of system components in the vessel are defined by the allocation of components to the volumes in $V$. Component allocation is mathematically defined by matrix a $[k \times n]$ matrix $\zeta$, where $k$ is the number of installed system components and $n$ is the number of volumes represented in $V$. If component $i$ is allocated to volume $j$, $\zeta_{ij}=1$, otherwise $\zeta_{ij}=0$.   

Convert ship to network picture here.

 

\subsubsection{Distribution System Network} \label{sec:sysmodel}
Distribution systems between shipboard components can be thought of as a set of paths through the vessel configuration. The composite of these paths is a network that is necessarily a subset of $V$. However, there may be multiple types of distributions systems (i.e. electrical power or chill water) and their routings may be co-located throughout the vessel. Describing the connections within a single type of system and the interdependencies between systems due to shared routing paths requires a network of networks representation. A network of networks, or multiplex network, is defined as a set of $m$ network layers where each layer $l$ is an adjacency matrix of the same $n$ nodes. Within each layer, the edges between nodes are independent of the edges in other layers. For example, an edge would exist between a generator and a crane in the electrical power distribution layer, but would not exist in the chill water system distribution layer. 

Physical relationships within the vessel configuration define the possible routes for the distribution system between components. Thus, the nodes in the distribution system multiplex $D$ are the same nodes defined in the vessel configuration network $V$. Within multiplex layers, edges between nodes denote routing connections for the represented distribution system. In this paper, systems are modeled as directed, capacited network to capture resource flow. Each edge $D_{l,ij}$ in a multiplex layer $l$ is weighted with capacity $c$, $c \in C_l$, defining maximum resource flow from node $i$ to node $j$, where $C_l$ is a set of positive integers representing the possible installed capacities for system $l$. Components allocated within the vessel are flow sources and sinks, denoted by their demand $d$. The demand of component $i$ in system layer $l$ is denoted by $d_{l,i}$. Following conventional notation, sinks have positive demand and sources have negative demand. Figure (here) demonstrates a multiplex network for two distribution systems.

demonstration



\subsection{ACO} \label{sec:ACO}
Ant Colony Optimization (ACO) is an evolutionary meta-heuristic optimization scheme inspired by the foraging behavior of ants. Originally applied to the Traveling Salesman Problem \citep{Dorigo1996}, ACO has since been successfully employed in many instances of combinatorial optimization problems (\cite{Dorigo1999} provides an overview). The ACO approach uses colonies of ants which explore a solution space and deposit artificial pheromones to guide future ants. Ants who find strong solutions deposit relatively larger amounts of pheromone, positively reinforcing its solution and increasing the local search power. The simple mechanics of AS create a powerful optimization algorithm that guides solution design while preventing premature convergence. 

The original ACO algorithm was the Ant System (AS) approach \citep{Dorigo1996} which had ants probabilistically follow a pheromone trail along a network. An ant at node $i$ would chose to move to a neighboring node $j$ using on a transition rule based on the amount of pheromone $\tau_{ij}$ between the node pair. The transition rule, defined as 

\begin{equation}
p_{ij}=\frac{\tau_{ij}^\alpha\cdot\eta_{ij}^\beta}{\sum_{h\in\Omega}
	\tau_{ih}^\alpha\cdot\eta_{ih}^\beta}
\label{eq:AStrans}
\end{equation}

\noindent where $\eta_{ih}$ is some heuristic information about the preference of node $h$, $\alpha$ and $\beta$ are the influence factors of the pheromones and heuristic, and $\Omega$ is the set of neighbors of node $i$. After each ant created a solution to the TSP, pheromones were updated as

\begin{equation}
\tau_{ij}=(1-\rho)\cdot\tau_{ij}+\sum_{k=1}^{A}
\Delta\tau_{ij}^k
\label{eq:ASup}
\end{equation}

\noindent where $\rho$ is the evaporation pheromone evaporation rate, $A$ is the total number of ants, and $\Delta\tau_{ij}^k$ is the amount of pheromone ant $k$ deposits based on the quality of its solution. After the pheromones are updated, the next generation of ants are started, using the new pheromones to guide their transition rules. Ant generations are iterated, progressively reinforcing successful paths, until a convergence criteria or computation time is met.

\subsection{ACO Extensions}

Since its inception, significant extensions have been made to the original AS algorithm in order to improve solution quality and add multi-objective optimization functionality. Variants such as elitist solution updating (elitist ants), pheromone constraints \citep{Stutzle2000}(add hypercube), and local search and reinitialization \citep{Ashraf2013} aim to improve solution quality and search capability. Multi-objective Ant Colony Optimization (MOACO) extensions such as \citep{Angus2009,Iredi2001} use combinations of multiple heuristics and pheromone trials to encourage ants to explore solutions along a Pareto front. \cite{Garcia-Martinez2007} and \cite{Rada-Vilela2013} provide extensive reviews of MOACO method and applications. 

The proposed SoS design method is a bi-objective optimization between survivability and representative cost. Following the approach used by \cite{Iredi2001}, pheromones for each objective are encoded in $M=(\tau_{ij})$ and $M^{'}=(\tau_{ij}^{'})$ respectively. Solution search is guided along the Pareto front by separating $m$ ants into $p$ colonies of $m/p$ ants and creating composite pheromones that interpolate between objectives. In the most straight-forward implementation, each ant $k$, $k \in [1,m/p]$ in a colony uses weighting factor $\lambda_{k}=\frac{k-1}{m/p-1}$ to determine the influence of each objective. More complicated weighting rules create disjoint or overlapping $\lambda$-intervals between colonies. Regardless of weighting, the transition rule (\ref{eq:AStrans}) is modified to

\begin{equation}
p_{ij}=\frac{\tau_{ij}^{\lambda_k\alpha}\cdot\tau_{ij}^{'(1-\lambda_k)\alpha}\cdot\eta_{ij}^{\lambda_k\beta}\cdot\eta_{ij}^{'(1-\lambda_k)\beta}}{\sum_{h\in\Omega}\tau_{ih}^{\lambda_k\alpha}\cdot\tau_{ih}^{'(1-\lambda_k)\alpha}\cdot\eta_{ih}^{\lambda_k\beta}\cdot\eta_{ih}^{'(1-\lambda_k)\beta}}
\label{eq:Bitrans}
\end{equation}

\noindent where $\eta_{ij}$ and $\eta^{'}_{ij}$ are heuristic information for the first and second objectives respectively. It should be noted that each colony $p$ has separate pheromones $M_p$ and $M^{'}_p$ which ants from that colony use in Equation \ref{eq:Bitrans}. \cite{Iredi2001} proposed two pheromone update rules of this method: update by origin and update by region of non-dominated front. The former updates only an ant's colony of origin with solution pheromones. The latter updates a colony's pheromones based on a solution's location on the non-dominated front, in effect making colonies responsible for specific solution spaces. The following section will apply elements of these extensions to a create an bi-objective ant colony SoS design exploration method.

\section{SoS ACO Methodology}
System of systems design requires significant alteration and extension of the existing ACO methods. The largest alteration is that solutions are composed of multiple interdependent networks instead of the single network solutions of classic approaches. SoS solution generation thus requires each ant to create a network of each distribution system's paths through $V$. In the proposed method, for each system $i=1, \ldots, l$, an ant creates the system network $D_i$ independently of the other systems. The composite multiplex network $D$ is then permuted in a localized search method and initial and permuted solutions are evaluated to quantify the performance of the SoS solutions. After a full generation of ants have been generated and evaluated, solutions on the Pareto front are used to update the pheromone matrices for the next ACO generation. This process is described in Figure (here) and the remainder of this section will detail its execution and describe a naval survivability demonstration.  

Process




\subsection{SoS Generation} \label{sec:gen}


Complications in solution generation arise from differences in SoS architecture design and the TSP network traversal problem. First, SoS's are not well suited for design by a single ant. They often include multiple sources and sinks which must be connected through different system types. This connectivity is directed - sources supply sinks - and branching meaning a lone ant would be unable to create a system without significant network traversal and looping. To address this, the proposed method allows an ant to create multiple branches each time an ant leaves a node. This method has been employed in multi-path routing of wireless sensor networks by \cite{Yang2010}, but is significantly expanded in this application. For each node $i$ in the network containing the SoS, a set of branching pheromones $B$ are stored for the number of branches $b$ an ant may create where $b \in [1,\mathcal{N}_i]$ and $\mathcal{N}_i$ is the number of neighbors of $i$. An ant at node $i$ then choses how many branches $b$ to create using the following transition rule,

\begin{equation}
p_{ib}=\frac{B_{ib}^{\lambda_k\alpha}\cdot B_{ib}^{'(1-\lambda_k)\alpha}}{\sum_{h\in\mathcal{N}_i}B_{ih}^{\lambda_k\alpha}\cdot B_{ih}^{'(1-\lambda_k)\alpha}}.
\label{eq:Branch}
\end{equation}

\noindent Note that unlike the original implementation of bi-criterion ACO, (\ref{eq:Branch}) does not use a heuristic. In testing, no heuristic improved the solution quality or convergence behavior of the algorithm. 

Once the number of branches is chosen, $b$ directed edges are added to the solution network by applying (\ref{eq:Bitrans}) $b$ times. In Algorithm \ref{al:Nstruct} this process is called antBranch($i$, antID) where antID defines the ant number and colony for calculating $\lambda_k$. Each time (\ref{eq:Bitrans}) is applied, an edge from $i$ to the chosen neighbor $j$ is added to the network $G^\psi$ which defines the structure of system $\psi$. $j$ then removed from the subsequent transition calculations, ensuring that $b$ branches are added in total. In Algorithm \ref{al:Nstruct} this process, adding $b$ directed edges from $i$, is called antNode($i$, $b$, antID) which returns a list of nodes chosen. This process, is repeated for each active branch in a solution generation step, enabling an ant to create complex branching and converging system structures. 

Introducing branching behavior creates difficulty with the termination criteria for an ants solutions. Classic applications to TSP consider a solution completer once all nodes in a network are covered. In SoS design, it may be undesirable to cover all possible nodes due to increased routing cost or the creation of unnecessary system interdependencies. Avoiding this requires a new termination criteria. An obvious approach is to terminate branches of the solution once they encounter a system sink. However, this prevents routing sinks in serial and artificially limits solution freedom. To avoid this, the proposed method uses a consensus-based approach to termination: once all system sinks are included in the solution, branches at node $i$ chose to termination outcome $t$, $t\in[0,1]$ following the transition rule,

\begin{equation}
p_{it}=\frac{T_{it}^{\lambda_k\alpha}\cdot T_{it}^{'(1-\lambda_k)\alpha}\cdot\zeta_{it}^{\beta}}{\sum_{h\in[0,1]}T_{ih}^{\lambda_k\alpha}\cdot T_{ih}^{'(1-\lambda_k)\alpha}\cdot\zeta_{ih}^\beta}
\label{eq:Term}
\end{equation}

\noindent where termination pheromones for $i$ are stored in $T$ and $\zeta$ is the termination heuristic. Probabilistically choosing $t=0$ continues the branch and $t=1$ terminates the branch at node $i$, preventing it from moving to any other nodes. Note that because, branches are prevented from terminating unless all sinks are included in the network, $p_{i1}=0$ until that condition is met. This process, antTermination($i$, antID) causes solution generation to conclude once all branches have been terminated. These steps are combined with the standard ACO transition rule to generate network structures for each system in the SoS. Networks are generated with Algorithm \ref{al:Nstruct} which is initialized by setting a branch at each source node $s$ in system $\psi$. The final step in Algorithm \ref{al:Nstruct}, pruneGraph(), removes branches from graph that do not contribute to a path between sources and sinks. These extra branches are either a remainder from the termination process or small off-shoots paths that go to a non-source, non-sink node and return following the same path. 

Creating each system network can be split into two steps -  network design and capacity design - which are explained in Section \ref{sec:gen}. 

feasible systems, decision rules, capacity.


\subsection{System Generation} \label{sec:trans}
Generating solutions to an SoS design problem is more complicated than traditional solution to the TSP. First, SoS's are not well suited for design by a single ant. They often include multiple sources and sinks which must be connected through different system types. This connectivity is directed and capacitated - sources supply sinks through flows - and may be connected by multiple branches of the distribution system. This means a lone ant would be unable to create a system without significant network traversal and looping. To address this, the proposed method allows an ant to create multiple branches each time an ant leaves a node. This method has been employed in multi-path routing of wireless sensor networks by \cite{Yang2010}, but is significantly expanded in this application. 

Incorporating branching walks into solution generation requires additional ant behaviors and associated transition rules. Unlike generating solutions to the TSP, where the only decision involved is which edge to traverse next, this method employs three decision types. The first decision occurs at the start of every step of the SoS generation process. Each ant in the walk uses a branching pheromone to select the number of branches it will create. Then for the number of branches selected, directed edges are added to the solution through an iterative application of the traditional ACO transition process. Finally after some termination criteria is met, the ant adds a flow capacity to each edge in the solution based on a third transition rule.

Each transition has a unique pheromone type, but uses the bi-criterion transition rule from \cite{Iredi2001}. XXX describes a generalized decision process, XXX, which takes a set of pheromones and heuristics and probabilistically returns a decision based on Equation \ref{eq:Bitrans}. XXX also incorporates a dissipation factor X which reduces the pheromone in P corresponding to the decision process output for the remainder of the ants walk. This mimics ants ``eating" the pheromone trail as they pass through it as seen in (cite ACS). In testing, the additional dissipation factor has been effective at preventing excessive cycling behavior during solution generation.

Algorithm here

Using XXX, Figure XXX describes the random walk process for creating single systems of the SoS. The penultimate step in the process, pruning, removes branches from graph that do not contribute to a path between sources and sinks. These extra branches are either a remainder from the termination process or small off-shoots paths that go to a non-source, non-sink node and return following the same path. The final step in the process checks the flow feasibility of the design, that the sink demands for the system are satisfied based on minimum cost flow.

Walk here


After the walk is completed, each edge in the solution is allocated a capacity through,

Capacity decision

where 

Creating a solution to a SoS design problem requires implementing this process for each system in the SoS. The resulting multiplex $D$ is a single ant's solution which can then permuted in local search and evaluated in the pheromone update process.


\subsection{Local Search} \label{sec:search}

ACO heuristic methods provide a coarse-grained search of a solution space, this is often augmented with a local optimizer to help refine the search (cite). Local search use the solution developed by an ant and attempt to improve it through methods ranging from genetic algorithms, to linear programming, to heuristic solvers. In this work, the local search is used to change pieces of the SoS networks with a crossover mutation scheme. This section will describe generally how local search is included in the SoS design algorithm. The exact local search algorithm used in a SoS design problem is dependent on the objectives being optimized. Later in this paper, an example of local search for minimum cost and maximum survivability is provided. 

After an ant completes the SoS generation process, its solution $D$ acts as the parent for the search method. Depending on the employed method, the local search will create one or more child solutions. In the simplest process, the parent and child solutions are then added to the solution pool for evaluation and pheromone update. However, some local search methods may force premature convergence of the algorithm or limit the ACO exploration power. To mitigate this, we introduce a crossover that treats the child solutions as possible mutations to the parent solution.

Crossover is evaluated by parent-child pairs. First the network differences between the parent and child are identified. Then for each difference, the network element from the child solution is accepted with probability $p_c$. This creates a mixed SoS solution between the parent and child whose level of mixing is controlled by the definition of $p_c$. At extremes, $p_c=0.0$ or $p_c=1.0$ the mutation is either fully the parent or fully the child, respectively. The level of crossover may be uniform or function of the algorithm progression. For example, the $p_c$ can start at $0.0$ in the first generation of the ACO and increase as the algorithm progresses.

Crossover on networks introduces connectivity issues because the accepted elements of the child may not be connected to elements of the parent. To prevent disconnect or extraneous edges and nodes, after the crossover process is completed each network in the SoS solution is pruned with the same process described in Section \ref{sec:trans}. Following crossover, the original parent and mutated solutions are evaluated by the objective function and added to the solution pool.


\subsection{Pheromone Update}
The final step in the ACO process is updating the pheremone trials of each ant. However, the extensions that have been added to this method, complicate the application of (\ref{eq:ASup}). Before discussing the mechanics of the pheromone update, the exact structure of the pheromones used in the algorithm should be clarified. Because each system $i=1\ldots l$ are designed independently, each has a separate pheromone structure used in the associated transition rules. Extending to incorporate bi-objective optimization requires using $P^i$ and $P'^{i}$ which are data structures containing all pheromone information for the first and second objective respectively. Further, following the approach in \cite{Iredi2001} each colony $p$ has unique pheromone structure $P^l_p$ and $P'^{l}_p$ for each system. Ants from colony $p$ use their respective structures during their generation process.  

The proposed method adapts the generic update rule with a normalized quality function and elitist pheromone updates. Colony's pheromone structures are only updated by ants from that colony which increasing selection pressure and encourages search in less dense regions of the non-dominated front \citep{Iredi2001}. These adaptations provide two benefits: Normalization allows pheromones to be updated relative to their solution quality while maintaining stable pheromone scale \citep{Blum2004}. Elitist updating accelerates solution convergence and improves solution quality by only allowing non-dominated solutions in a generation to update their pheromones \citep{Iredi2001}. 

After each ant in the generation creates feasible SoS solutions, solutions are evaluated by the objective functions. Non-dominated solutions $s_{upd}$ are selected to update their pheromone trails, where $s_{upd,i}=[s_i,s'_i]$ and $s_i$ and $s'_i$ are the fitness for the first and second objective, respectively. Solution scores are normalized to the global best objective score $s_{best}$ and $s'_{best}$ by the quality functions $F(s)$ and $F'(s')$. $F(s)$ shows the quality function for a maximization and $F'(s)$ for a minimization.

\begin{equation}
\begin{split}
F(s)=\frac{s}{s_{best}}\\
F'(s')=\frac{s'_{best}}{s'}.
\end{split}
\label{eq:Qual}
\end{equation}

The pheromone update increments, $\theta_i$ and $\theta'_i$, for ant $i$ with fitness $[s_i,s'_i]$, are normalized to the total quality of all updating solutions. Pheromones increments are defined as,

\begin{equation}
\begin{split}
\theta_i=\frac{F(s_i)}{\sum_{a\in s_{upd}}^{}F(a)}\\
\theta'_i=\frac{F'(s'_i)}{\sum_{a\in s_{upd}}^{}F'(a)}.
\end{split}
\label{eq:Inc}
\end{equation}

The original pheromone update, (\ref{eq:ASup}), can be generalized to any pheromone in colony $p$, as

\begin{equation}
P_p \leftarrow (1-\rho)P_p+\rho\sum_{s \in s_{upd}}^{}\theta_s
\label{eq:Up}
\end{equation} 

\noindent where $\rho$ is the pheromone dissipation rate, $p_s$ is the colony of origin for the ant producing $s$, and $\delta$ is the Dirac delta function. 

Each pheromone in the pheromone structures are updated using (\ref{eq:Up}), increasing system traits that appear in the non-dominated solutions. For example, if an ant from colony $p$ has edge $D_{l,ij}$ with capacity $c$, $P^l_{p,ijc}$ and $P^{'l}_{p,ijc}$ are updated by $\theta$ and $\theta'$ respectively. 

The update procedure described in this section is conducted after every ant has created feasible solutions to the SoS. This completes the SoS ACO, described in XXX encapsulates pheromone updating and $S$ is the generation's solution pool. Convergence criteria can either by computation time, Pareto front characteristics, or based on some analysis of the pheromone structures (see \cite{Stutzle2000} for pheromone convergence analysis). 


\subsection{Application}

In naval vessels, sustaining operations while damaged, or survivability, is a critical aspect of distribution system design (cite here). While general design studies and guidelines provide suggestions for improving vessel survivability \citep{Doerry2007,Kok2012}, methods for optimizing the survivability of interdependent naval distribution systems are not currently available. The proposed ACO SoS design method will be used to demonstrated on a two system maximum survivability, minimum cost design study.

                                                               

\subsubsection{Survivability Fitness Function} \label{sec:surv}
The primary objective of naval distribution system design is to ensure the vessels mission capability in normal operating conditions and when damaged \cite{Amy2002}. Thus distribution systems must be designed to sustain operation under a variety of damage scenarios. This requires a method for prediction damage to vessel and the subsequent failures it may cause. However, damage is not limited to local area of an attack. Sustaining damage to the distributed system may trigger successive failure elsewhere in the system. This type of cascading failure is particularly damaging to interdependent networks \cite{Gao2011,Schneider2013} such as the naval distribution system of systems. Cascading failure prevention has been studied in general \cite{Schneider2011,Brummitt2012} and naval specific cases \cite{Trapp2015}. However, these studies are limited to more theoretical damage scenarios such as occupation probabilities or single edge loss. This section will detail a survivability fitness function to consider cascading failures in interdependent distribution system networks under a generalized damage model.

Distribution system survivability in naval vessels is inherently tied to the geometric locations of interdependent systems \citep{Amy2002}. When the vessel is damaged, components and distribution routings in the damaged area are isolated until their functionality is assessed. In this model, component functionality is tracked in a vector $F$ where $F_i=1$ if component $i$ is functional and $F_i=0$ otherwise. Components that are not functional have their demand set to $0$ in each of its corresponding elements of $d$. The vessel functionality $\psi$ is the percentage of functional components after damage is assessed. Given a distribution system network $D$ that is contained in the geometric vessel representation $V$, the initial damage the vessel can be modeled as:

\begin{enumerate}
	\item  Identify the network elements (node and edges) of $V$ corresponding to the area(s) that are damaged.
	\item For each damaged element in $V$, remove that element from the distribution system multiplex layers.
	\begin{enumerate}
		\item If the element is a node $V_{jj}$, remove node $D_{i,jj}$ and edges connected to that node $D_{i,\cdot j}$ and $D_{i,j \cdot}$ for multiplex layers $i=1\ldots l$. Components located at $V_{jj}$ are marked as not functional in $F$.
		\item If the element is an edge $V_{jk}$, remove edge $D_{i,jk}$ and it reciprocal edge $D_{i,kj}$ for multiplex layers $i=1\ldots l$.
	\end{enumerate}
\end{enumerate}


Initial damage can be defined in a number of ways to model different scenarios. For example the removal of a set of nodes and edges to represent specific hit, removal of a set of nodes and edges at a random location to represent a unique damage pattern, or probabilistic removal for incidental damage or more complex attacks. After initial damage is applied, the cascading effects are evaluated. In this fitness function, failure will be evaluated based on resource flow through the SoS network $D$ using their demand $d$. Flow demand satisfaction of each distribution system layer $D_i$ is evaluated for demands $d_i$ with a minimum cost flow algorithm. Every component whose demand in is not satisfied in any system layer, it is marked as not functional in $F$ and has its demand set to zero. If any components are found to be not functional, the flow evaluation process is run again, otherwise the process terminates and $\psi$ is calculated. 


In the presented design study, damage is applied as the uniform random removal of four nodes. The damage is chosen to demonstrate the possible applications of the survivability evaluation function, not model a specific scenario. Due to the random damage, survivability evaluated 35 times for each SoS design and the average functionality $\bar{\phi}$ is taken as the fitness value.


\subsubsection{Cost}

This study uses a cost based fitness function as a trade off to survivability. Without cost, the ACO forces solutions towards the most densely connected distribution systems as those tend to be more survivable. The resulting fully connected distribution system network is unlikely to represent a realistically feasible solutions due producibility concerns. However, a network based SoS producibility metric is not available so a simple cost metric is used as a surrogate to similar effect. The SoS cost fitness $\psi$ is the sum of capacities over all edges in an SoS $D$.



\subsubsection{Local Search}

The coarse-grained ACO search is refined with the local search and crossover method described in Section \ref{sec:search} using two child solutions - one attempting to optimize each fitness function. Crossover probability $p_c$ increases with the number of generations like

pc equation

After an ant generates a solution, the child solution for survivability is created by making every edge in $D$ bi-directional. If the edge from node $j$ to $k$ in layer $i$ exists, but the reciprocal edges $D_{i,kj}$ does not, $D_{i,kj}$ is added to the graph with the same capacity weighting as $D_{i,jk}$. The survivability local search creates potential flow reroutings without adding new network branches. More powerful survivability improvement methods exist in the literature, but the random damage cases make secondary optimization computationally expensive and not analytically tractable. Edge reciprocation is independent of the damage case, easily computed, and is effective for the presented study. 

The minimum cost child solution is simply the edges of $D$ used in a min-cost max-flow solution. The min-cost max-flow graph $f_i$ is computed for each multiplex layer $i$ using the respective component demands. If an edge in distributed system network has no flow in $f_i$, it is removed from $D_i$. While there may be many combination of min-cost max-flow solutions for a distribution system, this local search uses a single random solution. Once again, this is not the most elegant methods, but it is computationally efficient and is effective for the problem.


\subsubsection{Vessel and System}


\section{Results and Discussion}
Application results, why it matters

\section{Conclusion}
Broaden out


\section*{References}
\bibliographystyle{elsarticle-num} 
\bibliography{References}



%% else use the following coding to input the bibitems directly in the
%% TeX file.

%\begin{thebibliography}{00}
%
%%% \bibitem{label}
%%% Text of bibliographic item
%
%\bibitem{}
%
%\end{thebibliography}
\end{document}
\endinput
%%
%% End of file `elsarticle-template-num.tex'.
